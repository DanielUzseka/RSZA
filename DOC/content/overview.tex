%----------------------------------------------------------------------------
\chapter{Áttekintés}
\label{sec:overview}
%----------------------------------------------------------------------------
\section{A perifériaillesztő modul felépítése}
%----------------------------------------------------------------------------

% \begin{center}
%     \begin{tikzpicture}
%         \tikzset{
%             top/.style= {draw, rectangle, inner sep=20pt, minimum height=0.6\textwidth, minimum width=0.8\textwidth, align=center},
%             module/.style= {draw, rectangle, minimum height=0.5\textwidth ,minimum width=0.3\textwidth},
%             input/.style  = {coordinate},
%             output/.style = {coordinate}}
%         \node[input] (PCLK) at (-0.55\textwidth, 0.21875\textwidth) {};
%         \node[input] (PRESETn) at (-0.55\textwidth, 0.15625\textwidth) {};
%         \node[input] (PADDR) at (-0.55\textwidth, -0.09375\textwidth) {};
%         \node[input] (PSELx) at (-0.55\textwidth, 0.03125\textwidth) {};
%         \node[input] (PENABLE) at (-0.55\textwidth, -0.03125\textwidth) {};
%         \node[input] (PWRITE) at (-0.55\textwidth, -0.09375\textwidth) {};
%         \node[input] (PRDATA) at (-0.55\textwidth, -0.15625\textwidth) {};
%         \node[input] (PWDATA) at (-0.55\textwidth, -0.21875\textwidth) {};
%         \node[output] (SCL) at (0.55\textwidth, 0) {};
%         \node[module] (APB) at (-0.3\textwidth, 0) {};
%         \node[above left] at (APB.south east) {mod\_apb};
%         \draw[->] (PCLK) -- node[auto] {PCLK} (APB.west);
%         \draw[->] (PRESETn) -- node[auto] {PRESETn} (APB.west);
%         \draw[->] (PADDR) -- node[auto] {PADDR} (APB.west);
%         \draw[->] (PSELx) -- node[auto] {PSELx} (APB.west);
%         \draw[->] (PENABLE) -- node[auto] {PENABLE} (APB.west);
%         \draw[->] (PWRITE) -- node[auto] {PWRITE} (APB.west);
%         \draw[->] (PRDATA) -- node[auto] {PRDATA} (APB.west);
%         \node[module] (I2C) at (0.3\textwidth, 0) {};
%         \node[above left] at (I2C.south east) {mod\_i2c};
%         \draw[->] (I2C.east) -- node[auto] {SCL} (SCL);
%         \node[top, fit=(APB)(I2C)] (TOP) at (0,0) {};
%         \node[above left] at (TOP.south east) {mod\_top};
%         % \node[]
%     \end{tikzpicture}
% \end{center}
Mint az a \figref{modtop} ábrán is látszik, modulunk (\emph{mod\_top}) 2 almodult tartalmaz, a feladatkíírásnak megfelelően: egy \emph{mod\_apb} modulból, ami közvetlenül az APB buszra csatlakozik, és egy \emph{mod\_i2c} modulból, ami a soros kommunkáció vonalaival van összeköttetésben.

\begin{figure}[ht!]
    \includegraphics[width=\textwidth]{figures/overview}
    \caption{A modul magas szintű áttekintő ábrája.}
    \label{fig:modtop}
\end{figure}

Az APB modul a rendszerbusz vezérlőjeleit dekódolja, és továbbítja a szükséges adatokat az I2C modulnak.
Az I2C modul a rendelkezésére bocsátott adatokból lefolytatja soros kommunikációt.

A továbbiakban tekintsük át részletesebben az almodulok felépítését. A modulokhoz, és a későbbiekben a tesztek kapcsolódó forráskódok a dokumentum végén, a függelékben találhatóak.

\section{mod\_top}

\section{mod\_apb}
\subsubsection{Ki és bemenetek}
Mivel ez a modul közvetlenül az APB buszra csatlakozik, bemenetei megegyeznek a busz jeleivel, eltekintve a PREADY és PSLVERR jelektől, melyeket nem használunk. Továbbá tartalmaz egy darab 32 bites ki-bemeneti regisztert, amely tartalmazza az I2C modul számára küldött, illetve attól fogadott összes releváns adatot. A pontos tartalmat lásd az I2C modul tárgyalásánál.

\subsubsection{Reset}
    A modul szinkron resetet valósít meg, az órajel felfutó élére mintát vesz a PRESETn jelből, amelynek alacsony értéke mellett nullára állítja a belső állapotregiszterét, illetve az APB és I2C felé menő kimeneteit.

\subsubsection{Állapotok}
    Ez az almodul 4 belső állapotot különböztet meg az APB vezérlőjelek alapján, ahol \emph{X} az érdektelent (Don't Care), \emph{1}  a logikai magasat (illetve helyes címet), \emph{0}  pedig ennek ellenkezőjét jelöli. Az alábbi táblázat szemlélteti az állapotokat:\\[2ex]

    \begin{tabular}{l|c|c|c|c}
        \textbf{Állapot}& \textbf{PADDR} & \textbf{PSELx} & \textbf{PENABLE}   & \textbf{PWRITE}    \\
        \textbf{IDLE}   &   X            & 0              & 0                  & X                  \\
        \textbf{SETUP}  &   X            & 1              & 0                  & X                  \\
        \textbf{READ}   &   1            & 1              & 1                  & 0                  \\
        \textbf{WRITE}  &   1            & 1              & 1                  & 1
    \end{tabular}

\subsubsection{Logika}
    A vezérlőjelek dekódolása alapján, READ állapotban a ki-bemeneti regiszterét a PRDATA buszra kapuzza, WRITE állapotban a PWDATA busz tartalmát betölti ugyanebbe a regiszterbe.


\section{mod\_i2c}
\subsubsection{Ki és bemenetek}
    Ez a modul közvetlenül az I2C buszra csatlakozik, így rendelkezik egy kétirányú, háromállapotú SDA porttal, mely a nyitott-kollektoros működést valósítja meg, illetve egy SCL órajel kimenettel. Továbbá csatlakozik a mod\_apb modulhoz egy 32 bit széles regiszteren keresztül. Ez a regiszter tartalmaz minden információt a modul számára az I2C kommunikáció kezdeményezéséhez. A regiszter tartalmát lásd a \figref{reg} ábrán.

    \begin{figure}
        \centering
        \begin{tikzpicture}[scale=0.8]
            \bitrect{16}{32-\bit}
            \robits{0}{13}{}
            \rwbits{13}{3}{DATA[0:2]}
        \end{tikzpicture}\\[2ex]
        \begin{tikzpicture}[scale=0.8]
            \bitrect{16}{16-\bit}
            \rwbits{0}{5}{DATA[3:7]}
            \rwbits{5}{7}{ADDR[0:6]} % endianness??
            \rwbits{12}{1}{R/W}
            \rwbits{13}{1}{Sp}
            \rwbits{14}{1}{R}
            \rwbits{15}{1}{S}
        \end{tikzpicture}
        \caption{}
        \label{fig:reg}
    \end{figure}

\subsubsection{Reset}
    Az APB bus PRESETn jelének
\subsubsection{Állapotok}

\subsubsection{Logika}
