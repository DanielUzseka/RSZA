%--------------------------------------------------------------------------------------
% Elnevezések
%--------------------------------------------------------------------------------------
\newcommand{\bme}{Budapesti Műszaki és Gazdaságtudományi Egyetem}
\newcommand{\vik}{Villamosmérnöki és Informatikai Kar}

\newcommand{\bmemit}{Méréstechnika és Információs Rendszerek Tanszék}

\newcommand{\keszitette}{Készítette}
\newcommand{\konzulens}{Konzulens}

\newcommand{\bsc}{Szakdolgozat}
\newcommand{\msc}{Diplomaterv}
\newcommand{\bsconlab}{BSc Önálló laboratórium}
\newcommand{\msconlabi}{MSc Önálló laboratórium 1.}
\newcommand{\msconlabii}{MSc Önálló laboratórium 2.}

\newcommand{\pelda}{Példa}
\newcommand{\definicio}{Definíció}
\newcommand{\tetel}{Tétel}

\newcommand{\bevezetes}{Bevezetés}
\newcommand{\koszonetnyilvanitas}{Köszönetnyilvánítás}
\newcommand{\abrakjegyzeke}{Ábrák jegyzéke}
\newcommand{\tablazatokjegyzeke}{Táblázatok jegyzéke}
\newcommand{\irodalomjegyzek}{Irodalomjegyzék}
\newcommand{\fuggelek}{Függelék}

\newcommand{\szerzoA}{\vikszerzoAVezeteknev{} \vikszerzoAKeresztnev}
\newcommand{\szerzoB}{\vikszerzoBVezeteknev{} \vikszerzoBKeresztnev}
\newcommand{\szerzoC}{\vikszerzoCVezeteknev{} \vikszerzoCKeresztnev}

\newcommand{\selectthesislanguage}{\selecthungarian}

\bibliographystyle{huplain}

\def\lstlistingname{lista}

\newcommand{\appendixnumber}{6}  % a fofejezet-szamlalo az angol ABC 6. betuje (F) lesz
